%--------------------------------------------------------------------
%--------------------------------------------------------------------
% Formato para los talleres del curso de Métodos Computacionales tomado de una de los ejercicios
% Universidad de los Andes
%--------------------------------------------------------------------
%--------------------------------------------------------------------

\documentclass[11pt,letterpaper]{exam}
\usepackage[utf8]{inputenc}
\usepackage[spanish]{babel}
\usepackage{graphicx}
\usepackage{tabularx}
\usepackage[absolute]{textpos} % Para poner una imagen en posiciones arbitrarias
\usepackage{multirow}
\usepackage{float}
\usepackage{hyperref}
%\decimalpoint

\begin{document}
\begin{center}
{\Large  Tarea 4 Métodos Computacionales} \\
Juan Sebastian Martinez Acevedo - \textsc{201615516}\\
19 de Noviembre de 2018\\
\end{center}


\noindent

\section{Gráficas de la Ecuacion Diferencial Parcial - ODE}
\subsection{Gráficas a 45º grados}
\begin{figure}[H]
\centering
\includegraphics[scale = 0.6]{proyectil_1.png}
\caption{Gaficas a 45º grados}\label{Fi:Imag3}
\end{figure}

\subsection{Gráficas a varios grados}
\begin{figure}[H]
\centering
\includegraphics[scale = 0.6]{proyectil_2.png}
\caption{Gaficas a varios grados}\label{Fi:Imag3}
\end{figure}


\end{document}
