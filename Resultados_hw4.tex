%--------------------------------------------------------------------
%--------------------------------------------------------------------
% Formato para los talleres del curso de Métodos Computacionales tomado de una de los ejercicios realizados en clase
% Universidad de los Andes
%--------------------------------------------------------------------
%--------------------------------------------------------------------

\documentclass[11pt,letterpaper]{exam}
\usepackage[utf8]{inputenc}
\usepackage[spanish]{babel}
\usepackage{graphicx}
\usepackage{tabularx}
\usepackage[absolute]{textpos}
\usepackage{multirow}
\usepackage{float}
\usepackage{hyperref}


\begin{document}
\begin{center}
{\Large  Tarea 4 Métodos Computacionales} \\
Juan Sebastian Martinez Acevedo - \textsc{201615516}\\
19 de Noviembre de 2018\\
\end{center}


\noindent

\section{Gráficas de la Ecuacion Diferencial Ordinaria - ODE}
Las graficas para el proyectil evidenciaron un comportamiento anómalo al movimiento parabólico conocido en los cursos de física básica. Este comportamiento se denotó en su alcance y descenso desde la altura máxima donde no se completa una simetría con el lado inicial. Esto se justifica con la fuerza de rozamiento ejercida por el viento. 

Por otro lado, el análisis de los lanzamientos con condicines iniciales iguales pero con variación en el ángulo de lanzamiento, indicaron que la altura aumenta con el aumento del ángulo, mientras que el alcance horizontal se dio máximo a un ángulo de 20º. 

\subsection{Gráficas a 45º grados}
\begin{figure}[H]
\centering
\includegraphics[scale = 0.6]{proyectil_1.png}
\caption{Gaficas a 45º grados}\label{Fi:Imag3}
\end{figure}

\subsection{Gráficas a varios grados}
\begin{figure}[H]
\centering
\includegraphics[scale = 0.6]{proyectil_2.png}
\caption{Gaficas a varios grados}\label{Fi:Imag3}
\end{figure}

\section{Gráficas de la Ecuacion Diferencial Parcial - PDE}

En esta sección deberían ir las gráficas de la difusión en dos dimensiones. Debido a que no se lograron generar, se deja indicado sobre el archivo de texto. 


\end{document}
